% © 2025 Jeffrey Gewirtz
% Author: Jeffrey Gewirtz
% Independent Researcher in Theoretical and Mathematical Physics
% ORCID: 0009-0004-8500-0049
% License: Creative Commons Attribution 4.0 International (CC BY 4.0)
% Version: v11.2
% Prepared for peer review and public reproducibility verification.

\documentclass[12pt]{article}
\usepackage{amsmath}
\usepackage{graphicx}
\usepackage{hyperref}

\title{Resonant Substrate Hypothesis (RSH): A Universal Invariant Linking Information and Energy Dissipation}
\author{Jeffrey Gewirtz \\ Independent Researcher in Theoretical and Mathematical Physics \\ ORCID: 0009-0004-8500-0049}
\date{2025}

\begin{document}
\maketitle

\begin{abstract}
The Resonant Substrate Hypothesis (RSH) establishes an empirical invariant that links the temporal decay of information and energy in dissipative systems. Using gravitational-wave data, we measure the mutual information decay constant ($\gamma_{MI}$) and spectral energy decay constant ($\gamma_{spec}$), finding their ratio $F = \gamma_{MI} / (2\gamma_{spec}) \approx 1$. This suggests a fundamental symmetry between informational and energetic dissipation rates, positioning RSH as a potential conservation framework that unites thermodynamic and informational principles.
\end{abstract}

\section{Mathematical Formulation}
Mutual Information Decay:
\[ I(\tau) = I_0 e^{-\gamma_{MI}\tau} \]
Spectral Energy Decay:
\[ E(\tau) = E_0 e^{-\gamma_{spec}\tau} \]
Invariant Definition:
\[ F = \frac{\gamma_{MI}}{2\gamma_{spec}} \approx 1 \]

\section{Discussion}
The observed constancy of $F$ across multiple gravitational-wave events implies a universal relationship between the rates of information loss and energy dissipation. This provides a measurable foundation for time as correlation decay.

\section{License and Citation}
This manuscript version (v11.2) is released under the Creative Commons Attribution 4.0 International License (CC BY 4.0). When citing this work, please reference the Zenodo DOI associated with this version.

\end{document}
